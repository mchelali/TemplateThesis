\chapter*{Introduction}
%\startcontents[chapters]
\addcontentsline{toc}{chapter}{Introduction}  

Avec l’avènement de la technologie, l'imagerie satellitaire a pu progresser en lançant de nouvelles constellations afin d'observer régulièrement la planète avec une résolution spatiale moyenne, recouvrant une même zone géographique et à une haute fréquence temporelle. Par exemple, les capteurs SENTINEL-2 produisent des séries temporelles d’images satellites (STIS) avec une période de re-visite de 5 jours et une résolution spatiale de 10 à 20 mètres. 
Notre environnement est perpétuellement soumis à des changements dans l’espace et dans le temps, qu'ils soient naturels ou provoqués par l'Homme. L'imagerie satellitaire offre la possibilité de détecter et d'analyser ce changement, cartographier l'occupation du sol ou même prévoir les risques des désastres naturels.\\

Dans le but d'analyser et d'extraire de nouvelles connaissances à partir des données issues de ces constellations, l'ANR (Agence Nationale de Recherche) finance le projet TIMES (Exploitation de masses de données hétérogènes à haute fréquence temporelle pour l’analyse des changements environnementaux)\footnote{https://anr.fr/Projet-ANR-17-CE23-0015}. Ce projet est réalisé en partenariat entre quatre laboratoires de recherche : le laboratoire Image,
Ville, Environnement (LIVE) de l'Université de Strasbourg, notre laboratoire d'informatique (LIPADE) de l'Université de Paris, l'institut de recherche en informatique, mathématiques, automatiques et signal (IRIMAS) de l'Université de Haute-Alsace et le service régional de traitement d'image et de télédétection (SERTIT-ICUBE) de l'Université de Strasbourg.\\

Ma thèse s'inscrit dans le cadre de ce projet d'analyse de STIS. Le but est de produire de nouvelles connaissances qui permettent de caractériser la dynamique de l'environnement tout en tenant compte de l'information spatiale et temporelle. 
Dans ce contexte, nous nous intéressons à la définition d'une représentation spatio-temporelle à partir des séries temporelles d'images satellitaires et à l'utilisation de celle-ci pour distinguer différentes régions constituant le territoire (par exemple les zones urbaines, les zones agricoles) et l'identification des changements d'occupation des territoires (par exemple l'urbanisation, la déforestation). 
La disponibilité croissante de ces données temporelles permet de produire et de mettre à jour des cartes précises de la couverture terrestre d'un territoire.



