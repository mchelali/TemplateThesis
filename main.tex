\documentclass[12pt]{book}
%\usepackage{natbib}
\usepackage[
backend=bibtex,
style=numeric,
doi=false,
isbn=true,
url=false,
eprint=false,
abbreviate=false,
%sorting=none
]{biblatex}
\addbibresource{bibliographie}


\usepackage[french]{babel}
\usepackage[utf8]{inputenc}
\usepackage[T1]{fontenc}
\usepackage[includefoot, a4paper,left=3cm,right=2.5cm,top=3cm,bottom=3cm, twoside]{geometry}
\usepackage{xcolor}
\usepackage{stmaryrd}
\usepackage{amssymb}
\usepackage{amsmath}
\usepackage{libertine}
\usepackage[pdftex]{graphicx}
\usepackage{caption}
\usepackage{subcaption}
\usepackage{float}
\usepackage{multirow}
\usepackage{apalike}
\usepackage{minitoc}
\usepackage{tikz}
\usepackage{enumitem}
\usepackage{epigraph}
%\usetikzlibrary{shadows.blur}
\usepackage{lscape}
\usepackage{titletoc}
%\usepackage{lipsum}
%\usepackage{cite}
\usepackage{booktabs}
\usepackage{arydshln}
\usepackage[nonumberlist, nomain]{glossaries}
\usepackage{calc}
\usepackage[]{titlesec} 
\usepackage[export]{adjustbox}
\usepackage[bottom]{footmisc}
\usepackage{appendix}
\definecolor{linkColor}{HTML}{32a852}%black
\usepackage[colorlinks=true,
citecolor=linkColor,linkcolor=linkColor]{hyperref}

%\captionsetup{labelfont=sc}
\def\frenchtablename{Tableau}

\usepackage{fancyhdr}


\usepackage{lipsum}


\setcounter{secnumdepth}{3}

\interfootnotelinepenalty=10000


\newcommand{\maj}[1]{{\color{red!40!yellow}{{#1}}}}

%%%%%%%%%%%%%%%%%%%%%%%%%%%%%%%%%%%%%%%%%%%%%%%%%%%%%%%%%%%
\renewbibmacro{in:}{%
	\ifentrytype{article}
	{}
	{\bibstring{in}%
		\printunit{\intitlepunct}}}


\titleformat{\paragraph}
{\normalfont\normalsize\bfseries}{\theparagraph}{1em}{}
\titlespacing*{\paragraph}
{0pt}{3.25ex plus 1ex minus .2ex}{1.5ex plus .2ex}


%%%%%%%%%%%%%%%%%%%%%%%%%%%%%%%%%%%%%%%%%%%%%%%%%%%%%%%%%%%%%%%%%%%%%%%%%%

%\newcommand{\cm}[1]{\textcolor{blue}{#1}}

\definecolor{yourcolor}{HTML}{8a0e19}%{008bb2}

\titleformat{\chapter}[display]
{\normalfont\color{yourcolor}}
{\filleft\huge\color{black}\textsc\chaptertitlename\hspace*{2mm}%
	\begin{tikzpicture}[baseline={([yshift=-.1ex]current bounding box.center)}]
		\node[fill=yourcolor,circle,text=white] {\thechapter};
	\end{tikzpicture}
}
{1ex}
{\titlerule[1.5pt]\vspace*{1.ex}\Huge\color{black}\textsc}
[]

\titleformat{name=\chapter,numberless}[display]
{\normalfont\color{black}}
{}
{1ex}
{\Huge\textsc}
[]

%command to print the acutal minitoc
\newcommand{\printmyminitoc}[1]{%
	\noindent\hspace{1cm}%
	\colorlet{chpnumbercolor}{black}%
	\begin{tikzpicture}
		\node(s){
			\begin{minipage}{.9\linewidth}%minipage trick
				\printcontents[chapters]{}{1}{}
			\end{minipage}
		};
		{
			\color{yourcolor}
			\draw(s.north west)--(s.north east) (s.south west)--(s.south east);
		}
	\end{tikzpicture}
	\vspace*{2ex}
	
	#1
	%\vfill
	%\pagebreak
}

%%%%%%%%%%%%%%%%%%%%%%%%%%%%%%%%%%%%%%%%%%%%%%%%%%%%%%%%%%%%%%%%%%%%%%%%%%%%%


%\newcommand{\hsp}{\hspace{20pt}}
\newcommand{\HRule}{\rule{\linewidth}{0.7mm}}
\newcommand{\Hrule}{\rule{\linewidth}{0.3mm}}

%%%%%%%%%%%%%%%%%%%%%%%%%%%%%%%%%%%%%%%%%%%%%%%%%%%%%%%%%%%%%%%%%%%%%%%%%%%%%

\usepackage{tcolorbox}
\tcbuselibrary{breakable} 

\newtcolorbox{rmk}[1][]{
	breakable,
	%title=Remarque,
	colback=white,
	%colbacktitle=green!20!white,
	%left=1cm,
	left skip=1cm,
	coltitle=black,
	fonttitle=\bfseries,
	bottomrule=0pt,
	toprule=0pt,
	leftrule=4pt,
	rightrule=0pt,
	titlerule=0pt,
	arc=0pt,
	outer arc=0pt,
	colframe=yourcolor   
}	
%%%%%%%%%%%%%%%%% 			New commands			%%%%%%%%%%%%%%%%%%%%%%


\newcommand{\ie}{\emph{i.e.},~}
\newcommand{\eg}{\emph{e.g.},~}
\DeclareMathOperator*{\argmax}{arg\,max}


%%%%%%%%%%%%%%%%%% 		List of abreviations 		%%%%%%%%%%%%%%%%%%%%%%%%%
%\setacronymstyle{long-short} 
\newglossary{symbols}{sym}{sbl}{Liste des abréviations et symboles}
\makeglossaries

\newglossaryentry{SIT}{type=symbols,
	name=\ensuremath{STI},
	%sort=fn,
	description={Séquence Temporelle d'Images}
}

\newglossaryentry{SITS}{type=symbols,
	name=\ensuremath{STIS},
	%sort=fn,
	description={Séries Temporelles d'Images Satellitaires}
}




\begin{document}
	
	\begin{titlepage}
		\begin{center}
			\begin{tabular}{c@{\hskip 3cm}c}
				\includegraphics[height=1cm]{images/Universite_Paris_logo_horizontal_2000px.png} &
				\includegraphics[height=1cm]{images/LIPADE.png}\\
			\end{tabular}
		\end{center}
	
		
		\begin{center}
			{%\large
				Université de Paris\\
				École doctorale EDITE (130)\\
				Laboratoire d’Informatique Paris Descartes (EA 2517)
				%Équipe Systèmes Intelligents de Perception (SIP)
			}
			
			\vfill
			
			\HRule \\[0.1cm]
			{ \Large \bfseries Prise en compte de l'information spatiale et temporelle\\[0.05cm]pour l'analyse de séquences d'images }
			\Hrule \\
		\end{center}
		
		\vfill
		
		\begin{center}
			Présentée et soutenue publiquement le 26-11-2021 par :\\[0.3cm] 
			\textsc{\large Mohamed Tayeb Chelali}\\[1.5cm] 
			
			Thèse de doctorat en :\\[0.2cm]
			\textsc{\large Informatique }\\[.5cm]
			
			Spécialité :\\[0.2cm]
			\textsc{\large Imagerie}\\[1cm]
		\end{center}
		
		%	\vfill
		\vspace{1.5cm}
		
		
		%Devant un jury composé de %\\[0.1cm]
		\begin{center}
			Membres du jury\\[0.2cm]
			
			
			\resizebox{\linewidth}{!}{
				\begin{tabular}{llr}
					\textsc{FULLNAME}  &  \textsc{Grade} & Reporter\\
					\textsc{FULLNAME}  &  \textsc{Grade} & Reporter \\
					\textsc{FULLNAME}  &  \textsc{Grade} & Supervisor\\
					\textsc{FULLNAME}  &  \textsc{Grade} & Supervisor \\
				\end{tabular}
			}\\[1cm]
		\end{center}	
		
		\newpage
	\end{titlepage}



\pagestyle{fancy}

\fancyhead{}

\renewcommand{\chaptermark}[1]{\markboth{\textsc{#1}}{}}


\frontmatter
{
	\setlength{\parskip}{.7em}
	
	\titlespacing*{\section}{0pt}{.9em}{.8em}
	%\titlespacing*{\subsection}
	%{0pt}{5.5ex plus 1ex minus .2ex}{4.3ex plus .2ex}
	\renewcommand{\baselinestretch}{1.1}
	\sloppy
	
	\chapter*{Remerciments}
\addcontentsline{toc}{chapter}{Résumé} 

\lipsum[3]
	
}

{
	\hypersetup{linkcolor=black}
	
	{\tableofcontents}
	
	{
		\cleardoublepage
		\addcontentsline{toc}{chapter}{Liste des figures}
		\listoffigures
	}
	
	
	{
		\cleardoublepage
		\addcontentsline{toc}{chapter}{Liste des tableaux}
		\listoftables
	}
		
}

\printglossaries

\setlength{\parskip}{.7em}

\titlespacing*{\section}{0pt}{.9em}{.8em}
%\titlespacing*{\subsection}
%{0pt}{5.5ex plus 1ex minus .2ex}{4.3ex plus .2ex}
\renewcommand{\baselinestretch}{1.1}
\sloppy


\include{Chapters/absract}

\mainmatter

\fancyhead[RO]{\leftmark}
\fancyhead[LE]{\textsc{\chaptername~\thechapter}}


\chapter*{Introduction}
%\startcontents[chapters]
\addcontentsline{toc}{chapter}{Introduction}  

\lipsum[3]






\part{Stae of the art}
\chapter{Chapter 1}
\epigraph{\itshape  La volonté trouve, la liberté choisit. Trouver et choisir, c'est penser}{-- Victor Hugo}
{
	\hypersetup{linkcolor=black}
	\startcontents[chapters]
	\printmyminitoc{
	\maj{Pour créer un glossaire des acronyme, il faut citer les acronymes qui ont était déclarer dans le main.tex. 
		Ensuite, il faut lancer \textit{Makeglosaries} et builder une 2eme fois pour que la liste des acronyme apparait dans le pdf/ 
		une première citation \gls{SIT}.
		et voila une 2eme \gls{SITS}}
}

}


\section{section 1}



\lipsum[1] \cite{imagenet_cvpr09}

\lipsum[1] \cite{KrizhevskySH12}

\section{section 2}
\lipsum[2]

\subsection{subsection 1}
\lipsum[1]
\subsection{subsection 2}
\lipsum[1]

\part{Contribution}
\chapter{Chapter 2}
\epigraph{\itshape  La volonté trouve, la liberté choisit. Trouver et choisir, c'est penser}{-- Victor Hugo}

\startcontents[chapters]
\printmyminitoc{
	\lipsum[1]
}



\section{section 1}

\lipsum[1] \cite{imagenet_cvpr09}

\lipsum[1] \cite{KrizhevskySH12}

\section{section 2}
\lipsum[2]

\subsection{subsection 1}
\lipsum[1]
\subsection{subsection 2}
\lipsum[1]


\printbibliography
\addcontentsline{toc}{chapter}{Bibliographie}

\end{document}
